\documentclass[12pt]{article}
\setlength{\topmargin}{0in}
\setlength{\headheight}{1in}
\setlength{\headsep}{0in}

\setlength{\textheight}{8.0 true in}
\setlength{\oddsidemargin}{0in}
\setlength{\evensidemargin}{0in}
\setlength{\textwidth}{6.0 true in}

\title{Security in ad Hoc Networking \\
	       Team D\\
       Secure Group Communication - Design}

\author{
    Jeremy Balmos \\
    (\texttt{jcb7565@cs.rit.edu})\\[1ex]
\and
    Shawn Chasse \\
    (\texttt{smc0857@cs.rit.edu})\\[1ex]
\and
    Jeremy Dahlgren \\
    (\texttt{jad0883@cs.rit.edu})\\[1ex]
\and
    Eric Ferguson \\
    (\texttt{etf2954@cs.rit.edu})\\[1ex]
}


% Commands
\newcommand{\code}[1]{\texttt{#1}}

\begin{document}
  \maketitle
  \newpage
  \tableofcontents
  \newpage
    \section{Introduction}
    \label{xref:introduction}
	This paper introduces a implementation of a protocol suggested by
	Adrian Perrig in his \textit{Efficient Collaborative Key Management 
	    Protocols For Secure Autonomous Group Communication} paper.

     \subsection{Adrian Perrig Communication Paper Review}
     \label{xfer:adrian_perrig_communication_paper_review}
     	Adrian Perrig presented a paper on \textit{Efficient Collaborative
    	Key Management Protocols for Secure Autonomous Group Communication}. 
	In this paper he established three protocols for secure group 
	communication over any type of network media.\\
	\\
	Perrig described five basic properties that any \textit{group key 
	agreement protocol} (GKAP) should have.  Any GKAP should be 
	contributory, have implicit key authentication, have key integrity, 
	provide key confirmation, and provide key independence.  These
	properties provide perfect backward and forward secrecy.  For more
	information about these properties refer to the background section
	in his paper.\\
	\\
	Before presenting the protocols, Perrig stated several assumptions that
	the protocols will function under.  One assumption is that after a 
	group member has been authenticated it will not be malicious in any
	way.  Another assumption is that each group member trusts a
	Certificate Authority or Key Authentication Center being used.  A final
	assumption is that any group member fails by halting.\\
	\\
	The first protocol presented does have any type of user 
	authentication.  A binary tree is used to give structure to the group.
	Each group member is located at a leaf in the binary tree.  A group
	key is established by using a key generation procedure based on 
	Diffie-Hellman key agreement between nodes.  Keys for intermediate
	nodes are established by randomly selecting a node from the left 
	subtree and right subtree to execute the Diffie-Hellman key generation 
	procedure.  The key established at the root node is the group key that
	will be used for secure communication.  Since this protocol uses 
	Diffie-Hellman to generate a key, it is vulnerable to a man in the 
	middle attack.  Even though this protocol lacked group authentication, 
	it is still usable as a base for the other protocols to build upon.\\
        \\
	Next he presented a protocol, which incorporates user authentication.
	This authentication is generated through a trusted third party or a 
	certificate agency. The authentication is an authenticated 
	Diffie-Hellman scheme using digital signatures and a public key 
	infrastructure. The author went on to note that there is a 
	disadvantage to using this type of authentication. First, "... for each
	key agreement, the certificates need to be exchanged, which consumes
	bandwidth because of the large size of certificates." Furthermore 
	"... the signature of the PKI [public key infrastructure] needs to be 
	verified, which is computationally expensive." Based on this 
	disadvantage, the author decided to present a third and final 
	protocol.\\
	\\
	The third protocol provided in this paper is based on \textit{Gunther's
	identity based key agreement protocol}.  Gunther's authentication 
	strategy uses a key authentication center to distribute private and
	public information for each group member which is used in member
	authentication prior to any key agreement procedure.  
	The workings of Gunther's protocol can be found in Perrig's paper 
	and do not need to be repeated here.\\
	\\
	Our implementation of secure group communication is based on Perrig's 
	third proposed protocol.  The exact key agreement was changed because 
	of mathematical reasons.  We ended up using a Diffie-Hellman based 
	key agreement procedure which follows Perrig's algorithm for 
	generating the key up the binary tree. The inner workings of our 
	protocol implementation are described in the proceeding paper. Please 
	reference the attached copy of Adrian Perrig's paper if there are any 
	further questions about his protocols.
	\\
	
    \section{Design Notes}
    \label{xfer:design_notes}
     \begin{itemize}
       \item	A unique member in the group is called the \textit{Group 
		Controller (GC)}.  The GC is responsible for administration
		of group members as well as initializing/synchronizing the key 
		agreement procedure.
       \item    The \code{BinaryTree} class is an array based binary tree, see 
		section \ref{xref:binarytree}.
       \item    A group has both a secure channel, see section
		\ref{xref:securechannel}, for private group communication, as
		well as an insecure channel, see section 
		\ref{xref:insecurechannel}, for trivial group communication and
		management.
     \end{itemize}
    \section{Design Flaws}
    \label{xfer:design_flaws}
     \begin{itemize}
       \item    The initialization of the \code{SecureChannel} should be done
		within an additional thread.  This would free up the 
                \code{InsecureChannel} thread that is waiting on new messages, 
		so they are not lost.
       \item    Our lack of knowledge of M2MP prevented us from writing 
		protocols that effectively used its functionality.
     \end{itemize}
    \section{Known Problems}
    \label{xfer:known_problems}
     The following scenarios have not been handled yet.  In our ongoing attempt
     to provide a user with the functionality to communicate securely with more
     than one person, the following will be addressed.  At present, the outcome
     of the following scenarios will produce unexpected results.
     \begin{itemize}
	\item	A timeout occurs during the key agreement procedure
	\item	Two users attempt to create a group/join a group at the same 
		time.
	\item	A member joins the group during the key agreement procedure.
	\item	A member leaves the group during the key agreement procedure.
	\item   More than \begin{math} 2^{32} - 2 \end{math} members join the 
		group, since we keep track of members by their member index, 
		stored by an integer, and we reserver 0 and -1 as special cases.
		The integer is just incremented every time a member joins the 
		group, but leaving member's indexes are not remembered and 
		reused.
     \end{itemize}

    \section{Binary Tree}
    \label{xref:binary_tree}
     All members of the group are kept in the \code{BinaryTree}.  Every position
     in the tree is kept with a \code{BinaryTreeNode}.  
        \subsection{BinaryTree}
        \label{xref:binarytree}
	 The only resemblance this class has to a standard binary tree is that,
	 at most, any node is allowed to have 2 children.  The 
	 \code{BinaryTreeNode} objects are kept in an array.  By putting
	 the tree in an array, transmitting the structure of the existing tree, 
	 to newly joining members, was very easy.  As an array based
	 binary tree, Perrig's algorithm for key agreement was also much 
	 easier.\\
	 \\
	 When a new member is inserted into the tree, the tree tries to insert 
	 the new node as close to the root as possible.  Members can only be 
	 leaves, so every node that is not a leaf has to be an empty node with
	 the member index of -1, symbolizing that it's an intermediary node.
	 Also, anytime a new level is added, the entire level (breadth-wise) is
	 filled with place holders.  These place holders have a member index
	 of 0 and contain no data.\\
	 \\
	 The \code{BinaryTree} class also provides the functionality to store
	 a member's key based on level of choice.  That way the \code{Group}
	 can store the intermediate keys during the key agreement procedure

        \subsection{BinaryTreeNode}
        \label{xref:binarytreenode}
	 A \code{BinaryTreeNode} is nothing more than a wrapper that holds two
	 values.  A \\\code{java.math.BigInteger} that is used as the key for 
	 the current node and an \code{int} that is used as the member index.
	 The member indexes 0 and -1 are reserved for internal use.
	 \begin{description}
	   \item[0]	A placeholder.  There is no data here, but 
			\code{java.util.Vector} cannot contain \textit{holes}, 
			so in order to use it the way we wanted, placeholders 
			must be used.
	   \item[-1]    An intermediate node.  Because our members can only
			be in leaves, we need intermediary 
			\code{BinaryTreeNode}s on every level above the leaves.
			This is where keys are placed.  
	 \end{description}

    \section{Group}
    \label{xref:group}
     The CPU/brain of our Secure Group Communication package would 
     be the \code{Group} class.  All of the functionality that is provided by 
     the Secure Group Communication package is accessed, by the user, through 
     the \code{Group} class.  All of the other classes in this package
     are hidden from the user.\\
     \\
     The \code{Group} will work directly with the \code{InsecureChannel},
     see section \ref{xref:insecurechannel}, to carry out the setup
     of the group.  When a user requests to join a group, the \code{Group} will
     send out a packet looking for the Group Controller.  If the response times
     out the user will create the group.  If the user gets back a response
     from the GC, then the user knows that the group exists and that he is not 
     the GC.  The GC will also send the new user a list of the current members 
     and their respective positions in the tree, see section 
     \ref{xref:binary_tree}.

      \subsection{Group Public Signature}
      \label{xref:group_public_signature}
       The signature of \code{Group} is the signature of the Secure Group
       Communication, since all functionality is accessed through the 
       \code{Group} class.\\\\
	\begin{tabular} {ll}
	   \code{Group( String gn,} 
	     & Join/Create the group \textit{groupName}. \\
	   \hspace{.6in}\code{MessageReceived ref )} 
	     & \hspace{.3in} Pass yourself as a reference for \\
	     & \hspace{.3in} callbacks upon received \\
	     & \hspace{.3in} messages\\
	   \code{leave()} 
	     & Leave the group \\
	   \code{sendInsecureMessage( String msg )} 
	     & Send \textit{msg} over the \code{InsecureChannel} \\
	   \code{sendSecureMessage( String msg )} 
	     & Send \textit{msg} over the \code{SecureChannel} \\
	\end{tabular} 

      \subsection{Heartbeat}
      \label{xref:heartbeat}
	Because of an ad hoc environment's dynamic nature, you must check to see
	that every member in the group is still \textit{alive}, that is to say,
	they are still present.  We achieve the assurance, of all members
	being \textit{alive}, by having every member send a \code{Heartbeat}
	to every other member, every \code{HB\_delay} seconds (section 
	\ref{xref:group_performance_tuning}).  Every member then checks on
	the status of the other members every \code{HB\_check\_delay} seconds
	(section \ref{xref:group_performance_tuning}). If any member has been
	idle (i.e. we have not received a \code{Heartbeat} from them) in 
	\code{HB\_TO} seconds (section \ref{xref:group_performance_tuning}), 
	then we will notify the GC and remove that member from their 
	\code{BinaryTree}.

      \newpage
      \subsection{Group Performance Tuning}
      \label{xref:group_performance_tuning}
       Another constructor is available to the user to tailor the timeouts of
       the \code{Group} class. 

       \subsubsection{Constructor}
       \begin{tabbing}
       \code{Group(} \=\code{String} \hspace{.8in} \=\code{groupName, }      \\
		     \>\code{MessageReceived}      \>\code{applicationRef,}  \\ 
		     \>\code{int}                  \>\code{join\_TO,}        \\
		     \>\code{int}	           \>\code{HB\_send\_delay,} \\
		     \>\code{int}	           \>\code{HB\_check\_delay,}\\
		     \>\code{int}	           \>\code{HB\_TO,}          \\
		     \>\code{int}                  \>\code{key\_TO )}        \\
       \end{tabbing}
       \begin{tabular} {ll}
	   \code{join\_TO}
	     & Number of milliseconds to wait before join call timeout \\
	     & \hspace{.3in} and the group is created \\
	   \code{HB\_send\_delay}
	     & Number of milliseconds to wait in between heartbeat sends \\
	   \code{HB\_check\_delay}
	     & Number of milliseconds to wait in between checking the \\
	     & \hspace{.3in} status of other members \\
	   \code{HB\_TO}
	     & Number of milliseconds before assuming a group memeber \\
	     & \hspace{.3in} is dead and expelling them \\
	   \code{key\_TO}
	     & Time to wait before assuming a group member is dead \\
	     & \hspace{.3in} during key agreement \\
       \end{tabular}\\

       \subsubsection{Accessors and Mutators}
     All of these values, can be changed on the fly using the following 
     accessors and mutators. \\\\
       \begin{tabular} {ll}
	   \code{void setJoinTimeout( int t )}
	     & Change or acquire current \\
	   \code{int \hspace{.1in}getJoinTimeout()}
	     & \hspace{.3in}\code{join\_TO} \\\\
	   \code{void setHeartBeatDelay( int t )}
	     & Change or acquire current \\
	   \code{int \hspace{.1in}getHeartBeatDelay()}
	     & \hspace{.3in}\code{HB\_send\_delay}\\\\
	   \code{void setHeartBeatCheckDelay( int t )}
	     & Change or acquire current \\
	   \code{int \hspace{.1in}getHeartBeatCheckDelay()}
	     & \hspace{.3in}\code{HB\_check\_delay}\\\\
	   \code{void setHeartBeatTimeout( int t )}
	     & Change or acquire current \\
	   \code{int \hspace{.1in}getHeartBeatTimeout()}
	     & \hspace{.3in}\code{HB\_TO} \\\\
	   \code{void getKeyProcTimeout( int t )}
	     & Change or acquire current \\
	   \code{int \hspace{.1in}setKeyProcTimeout()}
	     & \hspace{.3in}\code{key\_TO} \\\\
       \end{tabular}

    \section{Communication Channels}
    \label{xref:communication_channels}
      \subsection{InsecureChannel}
      \label{xref:insecurechannel}
	The \code{InsecureChannel} is used for all group management.  When
	a user joins a group, since key agreement has not yet taken place, 
	the communication must be done insecurely.   After a user has 
	entered a group, then key agreement takes place, and is also done 
	over the \code{InsecureChannel}.  After key agreement has been 
	completed, all subsequent group communication is done over the 
	\code{SecureChannel}.
	\subsubsection{TimeoutException}
	\label{xref:timeoutexception}
	  The \code{TimeoutException} is thrown by the blocking 
	  \code{receiveMessage( long timeout, int address )} 
	  call in \code{InsecureChannel} after \textit{timeout} 
	  milliseconds.


    \subsection{SecureChannel}
    \label{xref:securechannel}
      All secure group communication is done over the \code{SecureChannel}.
      Once a group key has been established, through key agreement procedure
      carried out over the \code{InsecureChannel}, the \code{Group} class 
      initializes the \code{SecureChannel}.  The \code{SecureChannel} then 
      uses the \code{MultiKeyGenerator} to encrypt and decrypt messages as 
      needed.
		    
      \subsubsection{MultiKeyGenerator}
      \label{xref:multikeygenerator}
	The \code{MultiKeyGenerator} is used by the \code{SecureChannel}
	to encrypt and decrypt messages.  The \code{MutliKeyGenerator}
	uses password based encryption with MD5 and DES, see 
	\textbf{Java\texttrademark Cryptography Ex (JCE) Reference Guide}
	for more information.  The \code{MultiKeyGenerator} removes the 
	knowledge about how the data is encrypted from the 
	\code{SecureChannel}. Abstracting the data encryption and decryption
	from the \code{SecureChannel} is useful because it takes out 
	complexity from the channel, and also decouples the encryption from
	itself.

    \section{Future Work}
    \label{xref:future_work}
      \begin{itemize}
    	    \item   Change implementation of key agreement from current state
	    	    to use Gunther's key agreement. 
	    \item   Rework protocol structure so that it is optimized for speed.
	    \item   Create the SecureChannel in a separate thread so that it
	    	    will not block the group object from receiving insecure 
		    messages during key generation.
	    \item   Address the problem of timeouts during the key agreement
	    	    procedure.
	    \item   Manage the group member indexes so that numbers may be 
		    reused if the necessity arises.
	    \item   If a member leaves during key agreement, cancel key 
		    agreement and process leave.
	    \item   If a member joins during key agreement, cancel key agreement
		    and process join.
	    \item   Test our protocol in a real ad hoc environment.
      \end{itemize}
\end{document}
