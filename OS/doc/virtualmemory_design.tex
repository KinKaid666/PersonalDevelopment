  \chapter{Virtual Memory}
  \section{Introduction}
    This document explains the modifications made to NachOS to implement
    a virtual memory system.
  \section{Changes}
    \begin{itemize}
      \item When a page fault occurs, the exception handler
            calls \code{AddrSpace::TLB\_Miss()}.
      \item \code{AddrSpace} talks to \code{KSwap}, found in section 
	    \ref{xref:kswap}, for memory-related services.  This includes
	    allocations, deallocations, and requests for 
	    virtual-to-physical translations.  \code{KSwap} handles any
	    swapping that takes place using \code{SwapFile} objects.  Data relevant
	    to these services is stored in \code{struct}s which are, in 
	    turn, organized in lists within \code{KSwap}.
      \item For efficiency, we are using an extents tree 
      	    (See \ref{xref:swapfile}) rather than a \code{BitMap} to keep track 
	    of free space within the read-write swap-file.  Also, we are are
	    using a dual-keyed list for the reverse page-table entry.
	    (See \ref{xref:kswap})
      \item \code{AddrSpace} no longer has an array of 
	    \code{TranslationEntry} objects for all of its memory, but relies
	    on \code{KSwap} to supply memory.
      \item The machine keeps a tlb, or table look aside buffer, to hold cached
	    virtual to physical addresses.  Since these addresses are unique 
	    to each process, when a context switch is made, these must be 
	    invalidated.
    \end{itemize}
  \section{Conventions}
  \label{xref:convs}
    \begin{itemize}
      \item Class \code{KSwap} handles most portions of memory management.
      \item There is one sufficiently large read/write swap-file.
            Each process' executable file serves as a read-only
	    swap-file.  Each process has only these two files as
	    associated swap-files.
      \item The read/write swap-file has a fixed size.
      \item \code{KSwap} runs a kernel thread that copies
	    data to and from \code{SwapFile} objects.  It
	    does this based on asynchronous requests initiated by 
	    \code{KSwap::getPhysAddr()}.  
	    The requests are queued in \code{KSwap}.  The synchronization
	    of this is done by using a \code{Semaphore} variable in the
	    \code{KSwap} class and a \code{Semaphore} in the \code{AddrSpace}
	    class.
	    The daemon thread sleeps between requests.
    \end{itemize}

  \section{Memory Management}
    Since memory is no longer statically partitioned at the 
    start of a process, \code{KSwap} handles memory management
    instead of \code{AddrSpace}.\\

    \code{KSwap} allocates and frees memory with
    the public functions \code{allocMemForPID()} and \code{freeMemForPID()},
    respectively.
    The public function \code{KSwap::getPhysAddr()} checks
    if the translation exists in the reverse page table (list of all physical 
    memory and its associated process information) and returns the associated 
    physical address. 
    If the virtual address does not exist in the reverse page table,
    \code{KSwap} swaps the needed data into physical memory and returns the
    new address.  This happens asynchronously and is described in detail in
    section \ref{xref:kswap}.\\

    \code{KSwap} swaps memory to and from disk with the 
    help of class \code{SwapFile}. \code{KSwap} contains a list
    of all active \code{SwapFiles}.\\

  \section{AddrSpace Class}
    The class \code{AddrSpace} is no longer responsible for most memory
    management, and relies on \code{KSwap}.  The constructor for this class now
    makes use of \code{allocMemForPID()} in \code{KSwap}.  The destructor of
    this class makes use of \code{freeMemForPID()} in \code{KSwap}.  The 
    function \code{allocPhysMemPage()} and \code{freePhysMemPage()} in this 
    class are invalidated.

    The linear page table is no longer used and is replaced with
    an array of 4 \code{TranslationEntry} objects called the \code{TLB}. The
    \code{TLB} holds virtual addresses that have been translated to physical
    addresses and this is found in \code{machine}.
    
    The function \code{TLB\_Miss()} is called when a TLB entry does
    not exist for a given virtual address. This function is called from
    \code{ExceptionHandler} when a page fault occurs. \code{AddrSpace}
    obtains the physical address from the \code{KSwap} object using
    \code{getPhysAddr()}.  In \code{TLB\_Miss}, \code{AddrSpace} will 
    update the machine's TLB.\\

    Each \code{AddrSpace} object has a \code{Semaphore} that is used for
    asynchronous signaling when requesting a swap. See section \ref{xref:kswap}
    for more information.

  \section{KSwap Class}
  \label{xref:kswap}
    The \code{KSwap} class contains variables and functions for managing 
    memory.  At the start of NachOS, a kernel thread is spawned with
    the static function \code{KSwap::swapper()}.  This
    function blocks and waits for requests to swap data to and from 
    \code{SwapFiles}. A semaphore, \code{daemonThdSem\_}, is used for 
    this purpose.\\

    \code{KSwap} keeps all segment information in a \code{List} of
    \code{List}s of \code{SegTblEntry\_t} \code{struct}s 
    (See \ref{xref:structs}).  The outer list is keyed according to PID
    and the inner list is keyed according to the virtual address of the
    first page within that segment.\\

    Physical memory translations are stored in a \code{DKList} of
    \code{RevPageEntry\_t} \code{struct}s.  The list is keyed both
    on the physical page number (e.g the i'th physical page) and also
    on a packed "long long" that has both the virtual address for that
    page and the process' ID.\\

    A process' \code{AddrSpace} can allocate and free memory, with the use
    of \code{allocMemForPID()} and \code{freeMemForPID()}, and can request
    virtual-to-physical translations with the function \code{getPhysAddr()}.\\
    
    \code{KSwap::getPhysAddr()} looks in the reverse page-table to see
    if the translation exists. If it does, the function returns
    with the translated address.  If the translation does not exist, this
    indicates that the information is not in physical memory and needs to
    be swapped in. \code{KSwap::getPhysAddr()} wakes up the 
    daemon thread and causes the calling \code{AddrSpace} to wait using that 
    \code{AddrSpace}'s \code{Semaphore}. The
    daemon thread then swaps out the Least Recently Used page (if physical
    memory is full), and
    swaps in the necessary page using the segment table as a calculative reference.  
    Finally, the daemon thread wakes up the requesting thread so that the
    machine's TLB can be updated in \code{AddrSpace}'s \code{TLB\_Miss}.
    The kernel thread then sleeps and waits to service another
    request. \\

    The \code{KSwap} class swaps memory to and from files using
    \code{SwapFile} objects (See section \ref{xref:swapfile}).\\

  \section{SwapFile Class}
  \label{xref:swapfile}
    The \code{SwapFile} class abstracts the copying of data to and
    from files.\\
    
    There are two different cases for swap files.  There is the global system
    swap file and process executables.  In addition to storing important
    file information like file size, we need to note if a file is
    read-only. If a file is not read-only, a \code{ExtentInfo} object is used to
    keep track of what pages in the file are used.  The \code{BinTree} of
    \code{Extent}'s 
    (private to \code{SwapFile}) represents a tree of extents that efficiently
    maintains a list of free segments in the file.  Seek time for a space of
    a certain size approaches O(log(n)) rather than O(n) with a linear search
    in the \code{BitMap} class.\\
    
    \code{SwapFile::swapOut()} takes a physical address for the page being
    swapped out and a file position. \code{SwapFile::swapIn()} takes similar
    arguments.  All addresses are in bytes and should indicate the first byte
    of a segment.

  \section{Details of \code{struct}s Used in this Design}
  \label{xref:structs}
    \code{SegTblEntry\_t}:
        \begin{itemize}
	\item Range of virtual memory for segment
	\item Associated process ID
	\item (Boolean) Read-Only
	\item A bitmap to indicate which pages have been brought in at least
	      once.
	\item A bitmap to indicate which pages have been swapped out by the LRU
	      algorithm at least once.
	\end{itemize}
    \code{RevPageEntry}: (Inherits from NachOS' \code{TranslationEntry} class)
        \begin{itemize}
	\item (\code{TranslationEntry}) Physical page
	\item (\code{TranslationEntry}) Virtual page
	\item (\code{TranslationEntry}) (Boolean) Read-Only
	\item (\code{TranslationEntry}) (Boolean) Dirty
	\item (\code{TranslationEntry}) (Boolean) Used
	\item (\code{TranslationEntry}) (Boolean) Valid
	\item Associated process ID
	\end{itemize}
    \code{Request\_t}:
        \begin{itemize}
	\item Virtual Address
	\item Associated process ID
	\item A pointer to the semaphore in the calling \code{AddrSpace}
	\item A pointer to the return value to be passed back from 
	      \code{KSwap::getPhysAddr()}.
	\item A pointer to a boolean that will be assigned before returning
	      from \code{KSwap::getPhysAddr()} that indicates whether the 
	      page is read-only or not.
	\end{itemize}
    
    \section{Low Points}
	Our code is not as efficient as it could be. We have an unused variable
	\code{KSwap::curSegTable\_} that should contain the segment table for the
	currently running process. This \code{List} is set when a context switch
	occurs. This eliminates the need to look this up whenever a swap must
	occur.\\

	The free extent tree in \code{SwapFile} is a simple binary search tree.
	There is room for improvement in efficiency here. A B tree or AVL tree would
	allow for more efficient average and worst case performance.
	Halving the free extent until we obtain an extent that is
	size $s$ where $n <= s < n*2$ ($n$ is the length of the free extent)
	seems like a more efficient method since it yields a more balanced tree
	but then we have many small extents that can be combined into a larger
	extent. The largest segment request can now be $m/2$ where $m$ is the
	amount of memory free before allocation. If the extents are recombined we
	end up with the same result as splitting the larger extent once.\\

    \section{High Points}
	Our free extent list is much more space efficient and, looking at the
	implementation of \code{Bitmap}, is at least as time efficient as using a
	free page \code{Bitmap}. Also, this allowed us to work out some bugs in
	the free extent system which we will be using in our file-system
	implementation.\\

	Out page replacement algorithm is simple and quick. It is not the ideal
	replacement algorithm but it still yields very good results. Below are miss
	statistics for some sample user programs. The bigarray program accesses
	each element in an \code{int[2048]} array. First, it sets the values of
	each member, then accesses them again to force page reloads. The matmult
	and sort programs came with NachOS and are designed to stress virtual
	memory. Each program was run from the NachOS shell with a total of 32 pages of
	memory available. The first number is the actual number of that type of miss.
	The number in parenthesis is the percentage of total misses. The capacity
	misses vs. conflict misses is the ratio of capacity vs. conflict misses.\\

	\begin{tabular}{|c||c|c|c|c|c|}
	\hline
Program  & Cold         & Capacity         & Conflict       & Total & Cap vs Conf\\
\hline
matmult  & 46 (0.13\%)  & 30566  (84.04\%) & 5760 (15.84\%) & 36372  & 5.31    \\
sort     & 38 (0.00\%)  & 786918 (99.94\%) & 458  (0.06\%)  & 787414 & 1718.16 \\
bigarray & 71 (20.52\%) & 237    (68.50\%) & 37   (10.69\%) & 346    & 6.41    \\
\hline
	\end{tabular}
