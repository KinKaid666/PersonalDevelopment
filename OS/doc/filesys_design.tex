\chapter{File System}
\section{Introduction}
\label{xref:introFS}
  This document explains the changes made to NachOS to implement a file system.
  This file system will replace the wrapper functions to unix calls that
  were previously used.

\section{Changes}
\label{xref:changesFS}
  \begin{itemize}
    \item	We made the choice to not use the filehdr or directory 
		classes that were 
		provided to us, and we created our own structures. (See section
		\ref{xref:mtreeFS})
    \item	\code{NodeCache} and \code{DataCache} will be used to store
		active nodes and datablocks respectively and will be discussed
		in detail in sections \ref{xref:nodecacheFS} and 
		\ref{xref:datacacheFS}.
    \item	The functions in \code{OpenFile} will no longer be wrappers to
		the UNIX system calls, but will directly access our internal
		structures for holding the NachOS file system.
    \item	\code{ETree} will store all the free extents in our file system
		and will be described in more detail in section 
		\ref{xref:etreeFS}.
    \item	To stick with the standards of our file system and the standards
		of NachOS, new system calls were added.  See section 
		\ref{xref:syscallsFS} for more information.
  \end{itemize} 

\section{System Calls}
\label{xref:syscallsFS}
  The system calls added for file functionality return error codes as with
  the other system calls.  These are system calls that you would expect to see
  for any filesystem.
  \\
  \begin{itemize}
      \item	\code{int Stat( struct Stat\_t* stat, char* name, int nameSize )}
      \item	\code{int Chmod( char* name, int nameSize, T\_BYTE perms )}
      \item     \code{int Unlink( char* name, int nameSize )}
      \item     \code{int Mkdir( char* name, int nameSize )}
      \item     \code{int Move(char* name, int nameSize, char* newName, int newNameSize)}
      \item     \code{int Cd( char* name, int size)}
  \end{itemize}

\section{FileSystem}
\label{xref:filesystemFS}
  We are no longer using wrapper functions for the implementation of
  our file system, but we are keeping a similar class interface so as to make
  the merge much smoother.
  \\
  \begin{itemize}
      \item		\code{FileSystem::FileSystem()}
      \item		\code{FileSystem::\~{}FileSystem()}
      \item		\code{OpenFile* FileSystem::Open( char* name )} 
      \item		\code{bool FileSystem::Remove( char* name )}
  \end{itemize}
  These function were all modified to use the new underlying file system, 
  namely the \code{MTree*} classes.  Another change is now that 
  \code{FileSystem::Remove()} will return an \code{int} to give us better
  error handling.
  \\\\
  The following are new function call signatures and their usage:
  \\
  \begin{itemize}
      \item		\code{void FileSystem::format()}\\
	Formats the file system. 

      \item		\code{bool FileSystem::Create( char* name, int initialSize=0 )}\\
	Creates a file (UNIX creat), but we ignore the initalSize.
      \item		\code{bool FileSystem::mkdir( char* name )} \\
	Creates a directory (UNIX mkdir).

      \item		\code{MTreeMetaNode*  FileSystem::getMetaNode(const char* fname )}\\
	Gets the meta data node for file fname.
      \item		\code{MTreeInternalNode*  FileSystem::getFileNode(const char* fname )}\\
	Gets the internal node that heads a file subtree fname.
      \item		\code{MTreeInternalNode*  FileSystem::getFileNode(const char** comps, size\_t len )}\\
	Similar to the above function, but uses comps (preparsed absolute path)
	instead of fname.
      \item		\code{MTreeNode*  FileSystem::getNode( T\_WORD id )}\\
	Gets a node given the universally unique node id.
      \item		\code{void  FileSystem::putNode( MTreeNode* node, bool force=false, bool ensureInCache=true )}\\
	Writes a node to the file system.
      \item		\code{void  FileSystem::releaseNode( MTreeNode* node )}\\
	Releases the lock on the node, this must be called on any node 
	retrieved with getNode().
      \item		\code{unsigned char*  FileSystem::getData( T\_WORD id )}\\
	Gets a block of data given the universally unique block id.
      \item		\code{void  FileSystem::putData( T\_WORD id )}\\
	Writes a data block to the file system.
      \item		\code{void  FileSystem::releaseData( T\_WORD id )}\\
	Releases the lock on a data block, this must be called on any block
	retrieved with getData().
      \item		\code{T\_WORD  FileSystem::allocateNode()}\\
	Allocates a node on the file system.
      \item		\code{void  FileSystem::freeNode( T\_WORD id )}\\
	Frees a node on the file system.
      \item		\code{T\_WORD  FileSystem::allocateDataExtent( T\_WORD start, T\_WORD len~)}\\
	Allocates a data extent on the file system acording to the arguments
	passed
      \item		\code{void  FileSystem::freeDataExtent( T\_WORD start, T\_WORD len )}\\
	Frees a data extent that is reprsented by the arguments passed.
  \end{itemize}

  The idea behind \code{getNode()} or \code{getData()}, and their corresponding
  \code{releaseNode()} or \code{releaseData()} is to synchronize the access
  to the disc.

\section{OpenFile}
\label{xref:openfileFS}
  The interface for \code{OpenFile} is exactly the same as it was in the 
  projects previous to this.  This interface is unchanged, with the exception
  of returned error codes, making the migration to our file system much easier.
  \\\\
  Every function, however, was rewriten, as the underlying 
  data structure that represents our file system has changed considerably.

\section{MTree}
\label{xref:mtreeFS}
  \code{MTree} is the new class that makes up the underlying data structure
  that is the file system.  There are four primary types of nodes.  It is
  important to note that any object of MTreeNode type will be 32 bytes long.
  \begin{itemize}
    \item	\code{MTreeNode} is the superclass for all the nodes.  This
		contains one byte of header information.
    \item	\code{MtreeRawNode} contains 31 bytes of raw data.
		(e.g. filenames or free extent info)
    \item	\code{MtreeInternalNode} provides structure and 
		branchind. It contains up to four children and
		15 bytes of unformated data the can be used for additional
		information.
    \item	\code{MTreeMetaNode} stores information about a file/directory, such
		as filename and size.
    \item	\code{MtreeDataNode} contains to extents that point to actual
		file data nodes.
  \end{itemize}

  The basic structure is fairly intuitive.  Directories are an
  \code{MTreeInternalNode} with its first child pointing to an \code{MTreeMetaNode}.  
  This node contains information such as size, filename, and permissions.
  The directory but will be set. The remaining children point to file sub-trees.
  A file has the same structure, but its \code{MTreeMetaNode} does has the directory bit cleared.
  It's children are \code{MTreeDataNode}s.
  This restricts a directory to 3 children.  When all
  three children of a directory are in use, and a 4th is added, the last 
  child pointer is set to a new \code{MTreeInternalNode}. This node points to 
  the previous child the new child being added.
  This can be done with any level of indirection, allowing a number of children
  restricted only be available disk space.
  Due to time and complexity constraints, we only implement one level of indirection allowing
  for a total of 12 files per directory. \\
  Indirection in file sub-trees can increase the number of extents a file can have. We
  did not implement this indirection so a file is limited to 9 extents. This does
  not place limits on the maximum file size, only the maximum file fragmentation.
  See section \ref{xref:limitationsFS} for additional restrictions.

\section{ETree}
\label{xref:etreeFS}
  \code{ETree} is used to manage free entents.  The root node of the file system, having
  ID \code{0x0000}, has 2 pointers in its special section. These 
  point to two \code{MTreeRawNode}s which are root nodes for \code{ETree} objects.
  An \code{ETreeNode} is simply an abstraction of an \code{MTreeRawNode} and provides
  functions that store left and right children and extent starts and lengths. An \code{ETree}
  is a binary search tree containing free extents indexed on length. One \code{ETree} contains
  free node extents and the other contains free data block extents.\\

  The logic behind \code{ETree::putNode()} and \code{ETree::releaseNode()} is similar to
  \code{BinaryTree} in the \code{FileSystem} class.

\section{Cache}
\label{xref:cacheFS}
  \code{Cache} is the superclass for the node and data block caching syste.
  It implements the raw access to disk based on an ``address'' rather than a disk sector.
  \code{Cache} abstracts the disk an array of bytes. The parent class \code{Cache} also
  holds the caching data structure. The constructor of this class accepts
  the maximum size of the cache.\\
  \code{NodeCache} and \code{DataCache} implement the 
  remaining parts of a cache for caching nodes and data chunks respectively.
  Nodes are 32 bytes.  Data chunks are 128 bytes which corresponds to a sector size.
  Minimal changes are required to increase the data chunk size to a larger multiple
  of the sector size. The singleton \code{NodeCache} object is currently
  of size 64 and the data cache is of size 32.\\

  The parent class \code{Cache} has an abstract function that its children
  must implement.  This makes space in the cache.  It is a public 
  function to make \code{n} ``slots'' available in the cache.\\
  \code{virtual void makeSpaceFor( unsigned int numItems ) = 0}

\section{DataCache}
\label{xref:datacacheFS}
  \code{DataCache} is a subclass of \code{Cache}.  This class gets
  data chunks (sector sized) off the disk and caches them writing them
  back to disk when requested. This class keeps more than one client
  from accessing a given data block, or chunk at a given time.

  \begin{itemize}
      \item	\code{unsigned char* getChunk( T\_WORD id )}
      \item     \code{void giveUpChunk( T\_WORD id )}
      \item     \code{void freeChunk( T\_WORD id )}
      \item	\code{void writeOutChunk( T\_WORD id, bool force = false )}
  \end{itemize}

  These functions accept a chunk id which is unique.  The
  Boolean argument \code{force} in \code{writeOutChunk} will force an 
  immediate write to disk.  Otherwise, the data will be written to 
  disk when it is removed from cache.

\section{NodeCache}
\label{xref:nodecacheFS}
  \code{NodeCache} is another subclass of \code{Cache}.  
  This class will gets nodes off disk and caches them writing them
  back to disk when requested. This class prevents more than one client
  from accessing a node at a given time.

  \begin{itemize}
      \item	\code{MTreeNode* getNode( T\_WORD id )}
      \item     \code{void giveUpNode( MTreeNode* node )}
      \item	\code{void freeNode( T\_WORD id )}
      \item	\code{void writeOutNode( MTreeNode* node,} 
				 \code{bool force = false, }
				 \code{bool ensureInCache = true)}
  \end{itemize}

  These functions work similarlty to \code{DataCache}'s functions.  (See
  section \ref{xref:datacacheFS})  The addition argument to \code{writeOutNode},
  \code{ensureInCache} ensures that the item
  being written out will be in cache afterwards. This is used when adding a new node
  to then cache/disk.

\section{Structs}
\label{xref:structsFS}
    \code{Stat\_t}
    \\
    A struct that houses various file information.
    \\
    \begin{tabular}{|lll|}
    \hline
      \code{T\_BYTE} & \code{mode\_}      & The permissions of the file.     \\
      \code{char*}   & \code{filename\_}  & The name of the file.	     \\
      \code{bool}    & \code{directory\_} & Is this representing a directory?\\
      \code{T\_WORD} & \code{size\_}      & The current size of the file.    \\
      \code{T\_WORD} & \code{id\_}        & A unique node id.                \\
    \hline
    \end{tabular}
    \\\\
    \code{ChunkInfo\_t}
    \\
    This struct definition is private to the \code{DataCache} class.  It is
    used to keep track of info about data chunks in a \code{DataCache}
    object.
    \\
    \begin{tabular}{|lll|}
    \hline
      \code{unsigned char*} & \code{chunk\_}	& Reference chunk in memory \\
      \code{T\_WORD}        & \code{id\_}       & Unique id of chunk\\
      \code{T\_WORD}        & \code{diskAddr\_} & Disk address of chunk in bytes\\
      \code{bool}           & \code{inUse\_}	& Used to lock a chunk\\
      \code{bool}           & \code{dirty\_}    & Has data been modified?\\
    \hline
    \end{tabular}
    \\

\section{Limitations}
\label{xref:limitationsFS}
  \begin{itemize}
    \item	Filename length is limited to 123 characters. This limitation
		is dur to our implementation, not our design. A \code{MTreeMetaNode}
		contains up to four sub-trees for filename information. Our implementation
		allows only \code{MTreeRawNode}'s and not indirection \code{MTreeInternalNode}'s.
		This limits us to $4\cdot31$ characters including a trailing \code{NULL}.
    \item	File and directory children are limited to one level of
		indirection.  This is not a design limitation, but an implementation limit.
		Consequently, we can have only 12 files per directory.
    \item	When formating the system, NachOS will not allow you to 
		continue, but will exit after the the format.  This is the
		same for a UNIX to NachOS file copy.
    \item	When a filesystem is formatted, the sizes of the initial free node extent
		and free data extent are specified. This limites the maximum number
		of files/directories and the maximum amount of data. The file system design
		allows this distribution to be updated dynamically but the code
		has not been implemented. Currently, $\frac{1}{4}$ of the space is given to nodes
		and the remainder is given to data blocks.to do this is not implemented.
    \item	There is a significant amount of error handling that is not performed, mostly in userland
    		programs. This can cause strange behavior in certain circumstances. Due to
		time limitations, this error handling has not been implemented so care must be
		taken when performing certain operations.
    \item	Certain functions are buggy. Most of the code for the filesystem had been written
    		and tested but remains buggy. This is due to time limitations.
    \item	Some desired features have not been implemented due to time limitations. Most notably,
    		journaling support does not exist. The root node of the file system has room for a single
		data extent in its \code{special} section. This would point to a fixed length journal. All
		modifications to meta data would be written to this journal before they are performed
		and marked when completed. This allows for faster and more complete filesystem repair
		should the system crash. Since nachos is a userland file and crashes in the middle of writes
		are unlikely, this feature is unecessary.
    \item	We would like to reimplement \code{ETree} to use a balanced tree such as a B$^*$-tree,
		allowing for more efficient searches.
  \end{itemize}
