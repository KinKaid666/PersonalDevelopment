  \chapter{User Processes \& Multi-programming}
  \section{Introduction}
    This document explains the modifications made to NachOS to implement user
    processes and multi-programming features.
  
  \section{Nice Design Features}
    \begin{itemize}
      \item The exit path, although quite complex, handles the cleanup of fairly complex
	object interactions.
      
      \item Our stack allocation routines are easy to modify to accommodate dynamic
	stacks.
      
      \item \code{ThreadExit()} is called by all threads explicitly. This can be
	implemented with wrappers to \code{Fork()}. This flexibility allow for changes in
	the userland \code{Fork()} semantics without changing the kernel.
    \end{itemize}

  \section{Known Problems}
  \label{xref:knownproblems}
    \begin{itemize}
      \item Modifications to the \code{Thread} class can cause problems with
	\code{SWITCH()}. The current system is stable in this regard, but may not work if
	more is added to this class.
      
      \item The exit path is overly complex and utilizes a kernel thread to kill user
	threads. This sometimes causes synchronization issues, primarily if \code{Fork()}
	is called immediately before \code{Exit()}. A temporarily solution is to place a
	\code{for() Yield();} loop that yields some number of times (found by trial and
	error) before exiting. This give the \code{Fork()} enough time to establish a new
	thread.
      
      \item Many of the user programs are not bulletproof. The shell can not handle
	unnecessary whitespace.
	  
      \item There is an over-restriction on many of the \code{Process} data structures
	such as a maximum number of arguments.
      
      \item \code{ThreadExit()} must be called explicitly by all threads.
    \end{itemize}
  
  \section{Conventions}
    \begin{itemize}
      \item Kernel functions returning an integer error code indicate success
      with a zero (0) and failure with a function specific non-zero value.

      \item Kernel functions returning a pointer indicate failure by returning
      a \code{NULL} pointer. Actual error information is provided in a function
      specific manner, typically through use of \code{DEBUG()} statements.

      \item When a process has no more threads that are \code{READY-TO-RUN} or
      \code{BLOCKED}, the associated process is destroyed and its resources reclaimed.

      \item When a process is killed via an \code{Exit()} system call, all of it's
      threads are killed.
    \end{itemize}
  
  \section{Time Slicing \& Thread Overview}
    User threads are implemented using NachOS kernel threads. Therefore, most
    scheduling is handle by the existing mechanisms. The NachOS ``random
    yield'' feature is modified to support preemptive task switching. A machine
    timer is set to expire on a regular basis. The currently running thread's
    \code{Quantum} is decremented and a \code{Yield()} is forced when this \code{Quantum}
    reaches zero.\\

    User threads are kernel threads that are associated with a process, and thus an
    address space. One of the flaws in our design was to modify the \code{Thread} class
    instead of making a new class containing the \code{Thread} class. This may not be an
    OOP flaw but the \code{Thread} class is very touchy as to the size and positioning of
    member variables. The \code{SWITCH()} function is an assembly routine and needs to
    know exactly where stack information is located within the data structure. If we
    reorder some of the data we added to this class, nachos immediately core dumps when
    the first task switch is attempted. We found a configuration that seems to work but
    if we could do things over, we would change this portion of our design.\\

    We have added a thread ID, associated process ID, associated process pointer, and
    composite ID to the \code{Thread} class. The composite ID consists of a TID in bits
    0--7 and a PID in bits 8--15. This uniquely identifies the thread within the system
    and is primarily used as a user interface aid allowing a user to specify a thread
    using one number instead of two.\\

    The \code{Thread::join()} function allows a thread to wait for a process to finish. A
    condition variable in class \code{Process} is used to facilitate joining. See section
    \ref{xref:proc} for more information. We modified the \code{Thread::Finish()}
    routine. Previously, only the currently active thread can be finished. A global
    pointer is set to point to the current thread, then the thread sleeps, yielding the
    process. When the context switch occurs, the thread is unscheduled and destroyed. We
    needed the ability to finish an arbitrary thread so we changed \code{Finish()}. The
    thread being finished is set as the thread to be destroyed. If this thread is also
    the current thread, \code{Sleep()} is called. This is identical to the original
    implementation. If the thread being finished is not currently running, the
    \code{Finish()} function returns and the thread is destroy when the next context
    switch occurs. Only one thread can be destroyed at a time. We must ensure that the
    pending thread destruction is resolved before calling \code{Finish()} on another
    thread. This can be fixed with a mutex preventing access to the
    \code{threadToBeDestroyed} global variable if a destruction is pending; or the global
    \code{threadToBeDestroyed} could be replaced by a list of threads.\\
  
  \section{Memory Management}
    A paging memory system is used because there are many utilities supplied by NachOS
    which are conducive to paging, including hardware page table translation. When a new
    process is created it is allocated pages of memory for its code that will be set to
    read only. This is only possible if the code and data segments begin on page
    boundaries. The MIPS compiler suite used is set to align these sections on
    \textbf{code{0x80}} byte boundaries which is the currently size of a page in our
    system.\\

    Allocation of memory for a process is static. When a program is loaded, the required
    amount of memory is determined and an appropriate number of pages are allocated. A
    global \code{BitMap} keeps track of what pages of memory have been allocated.
    Additional space for thread stacks is also reserved. This means that there is a
    limit to the number of threads that can be active in a process at one time.
    Currently, the maximum is 4 threads. A \code{BitMap} in class \code{AddrSpace} keeps
    track of which pre-allocated stacks are in use. When a thread finishes, it's stack
    space becomes available for a new thread. Since the stacks are at the end of the
    process' address space, and an address space can utilize non-contiguous pages of
    memory, it is possible to dynamically allocate stack space. A modification to the
    \code{allocStack()} and \code{freeStack()} methods of \code{AddrSpace} is necessary
    to implement dynamic stacks.\\

    Functions to allocate physical memory and free allocated memory for an address space
    have been added. Also, two functions that copy data from user space to kernel space
    and back have been added. These are essentially the same as the \code{ReadMem()},
    \code{WriteMem()}, and \code{Translate()} functions in the MIPS simulator. The
    \code{copyArgs()} function places \code{Process} arguments on the specified thread
    stack. This is used to pass \code{Exec()} arguments to the initial thread of a
    process. Since the arguments are placed on the stack and the stack pointer is updated
    to reflect this, the initial thread of a process may have less stack space than other
    threads. When the thread is finished using the arguments, they may be poppedd off the
    stack to regain the lost space. See section \ref{xref:exec} for more
    information regarding arguments. The \code{vmstat()} function returns information
    about memory allocation and stack allocation. It will be used to implement a
    \code{vmstat()} system call during the next phase of the project. It is currently
    unused.\\

    The \code{AddrSpace} constructor needed modification to support paging. The default
    implementation assumed all virtual page numbers pointed to the equivalent physical
    page number. This limits you to one user process. When the page table is being
    initialized, the \code{allocPhysMemPage()} method is used to assign a physical memory
    page to a virtual memory page. Next, the newly allocated memory is zeroed. Instead of
    calling \code{bzero()} on the physical memory, an appropriately sized array is zeroed,
    then it is copied to the new user space with the \code{kernToUser()} function. An
    oversight that eluded us for quite some time involved the loading of the executable.
    We did not remember to rewrite this so that the data is loaded utilizing the page
    table translations so when a second process was exec'd, the data for that process
    overwrote the data for process zero. Now, the data is loaded into a temporary char
    array, then copied to user space with the \code{kernToUser()} method.\\
  
  \section{Process Class}
  \label{xref:proc}
    The \code{Process} class contains variables for storing process information as well
    as functions for manipulating these values. There are some process relevant constants
    defined as well.
    \begin{description}
      \item[MAX\_CHILD\_PROCS] Maximum number of active child processes that a process can
	have. This is an artificial restriction placed on a process due to memory
	constraints. This will probably be removed when virtual memory is implemented.
      \item[MAX\_FILE\_HANDLES] Maximum open files a process can have. This includes the
	two console I/O handles. This is also an artificial constraint. It is not
	strictly necessary right now but implements a certain level of security
	protection. For example, a process can not execute code to continuously open
	files and cause system overutilization.
      \item[MAX\_THREADS] This limits the number of active threads a process may have.
	This is used by \code{AddrSpace} to statically allocate stack space for threads.
	This will be unneeded if dynamic stack allocation is implemented.
      \item[MAX\_JOINS] This is another strictly unnecessary constraint that provides
	a reasonable limitation of the number of threads that can \code{Join()} on a
	process.
      \item[MAX\_ARGS] This is the maximum number of arguments that can be passed to
	\code{Exec()}. This is also a precautionary measure.
      \item[MAX\_ARG\_LENGTH] This is the maximum length of a single argument. See section
	\ref{xref:exec} for more information.
    \end{description}

    A \code{Process} contains a condition variable used for joining threads to processes.
    A thread requesting a \code{Join()} to the \code{Process} waits on the condition
    variable. When the \code{Process} exits, the condition variable is \code{Broadcast}
    releasing all threads joined.\\

    A \code{Process} also contains a list of threads, children, and file handles. These
    are dynamic \code{List}s. A \code{Process} can \code{exit()}, manage the above lists,
    open files, close file handles, \code{read()} and \code{write()} to file handles.
    Most of the \code{Process} class is self explanatory. The exit path is discussed in
    section \ref{xref:exitpath}.
  
  \section{ProcessManager Class}
    The \code{ProcessManager} is a singleton class that handles the interactions between
    \code{Process}' and the management of \code{Process}'. It contains a process table
    and the value of the next PID as well as accessors and mutators. The
    \code{getProcCount()}, \code{getProcInfo()}, \code{exit()}, \code{threadExit()},
    \code{exec()}, \code{join()}, and \code{fork()} system calls are implemented in this
    class.\\

    \code{ProcessManager} also contains the \code{initThread()} and \code{threadKiller()}
    static functions used to start and end threads. The \code{initThread()} function
    takes an \code{AddrSpace} pointer for the thread to begin. The \code{AddrSpace} is
    restored and the \code{Machine::Run()} function is called. This reentrant function
    begins execution of the thread's user code. The \code{PC} and \code{SP} must have
    been set before \code{Fork()} is called. The \code{threadKiller()} function is
    described in section \ref{xref:exitpath}.

  \section{SConsole}
  \label{xref:sconsole}
    The \code{SConsole} class allows for synchronous access to console I/O. It
    instantiates a \code{Console} and provides condition variable synchronization to the
    \code{Console} on the \code{read()}/\code{write()} level.

  \section{Other Class Changes}
    The following changes were made to existing classes and were not discussed in one of
    the above sections.

    \subsection{List}
      We added accessors for the first and last item of a \code{List}. This simplifies
      iteration.
    \subsection{BitMap}
      We added an accessor for the \code{numBits} member variable. This is used to obtain
      total memory page and total thread stack information for the
      \code{AddrSpace::vmstat()} function.

  \section{System Calls}
    The following section describes implementation details for the system calls
    implemented.

    \subsection{Halt}
      This system call executes \code{interrupt->Halt()} stopping the NachOS system.

    \subsection{Exit}
      This begins the exit path on the current process. See section \ref{xref:exitpath}
      for more information.

    \subsection{Exec}
    \label{xref:exec}
      The \code{Exec} system call establishes a new \code{Process} object and spawns the
      initial \code{Thread} for that \code{Process}. Arguments are passed via a modified
      \code{argc}/\code{argv} method. \code{argv} is a \code{char *} containing all of
      the arguments to the new \code{Process}' \code{main()}. Arguments are a fixed
      length within \code{argv} and are \code{NULL} padded. The length of \code{argv} can
      be calculated with $\code{argc} * \code{MAX\_ARG\_LENGTH}$. This approach was
      used, instead of traditional \code{char *argv[]} semantics, since it is very easy
      to determine the length of \code{argv}. If traditional semantics were used,
      multiple \code{userToKern()} calls would be required to locate the end of each
      argument string. Our approach allows for a single \code{userToKern()} call, and
      later a single \code{kernToUser()} call (for transferring \code{argv} to the new
      \code{Process}).

    \subsection{Join}
      The \code{Join()} system call is implemented in the \code{ProcessManager} class.

    \subsection{Create, Open \& Close}
      These system calls are implemented in the stub filesystem.

    \subsection{Read \& Write}
      These system calls are implemented in the \code{SConsole} class, described in
      section \ref{xref:sconsole}.

    \subsection{Fork}
      This starts a new thread in the current process.

    \subsection{Yield}
      This calls \code{Yield()} on the currently running thread.

    \subsection{ThreadExit}
      This causes the currently running thread to exit. See section \ref{xref:exitpath}
      for more information on the exit path.

    \subsection{Kill}
      This system call kills a \code{Process} identified by its PID. This was a call we
      added but did not need to use so it is not fully tested.

    \subsection{ProcInfo}
      This returns information about running processes including process name, PID,
      parent PID, number of threads, number of joins, and number of children. It is used
      to implement the \code{ps} user program. Future versions may include more specific
      thread information such as thread IDs.

    \subsection{GetProcCount}
      This returns the number of running processes.

    \subsection{VMStat}
      We did not finish this system call. It's primary use will be in the virtual memory
      section of NachOS so we did not bother completing this yet.

  \section{The Exit Path}
  \label{xref:exitpath}
    The exit path is a series of serial and parallel steps that cause threads and
    processes to exit. Different system calls and events may enter the path at various
    points. This section will describe the path from the point of view of the
    \code{Exit()} system call. This system call enters the path at the beginning so it
    provides a complete description.\\

    First, the \code{ProcessManager} removes the \code{Process} information from its data
    structures. If the process has a parent, the parent is notified to remove the process
    from its child list. If this is the last process, the system is halted.\\

    Next, \code{Process::exit} is called. Each thread is killed by the
    \code{ProcessManager}. Then, the joiners condition variable is \code{Broadcast}
    released.\\

    The \code{ProcessManager::threadKill} function unschedules the thread to be killed,
    calls \code{Finish()} on that thread, then \code{Yields()} to force a
    \code{SWITCH()}. This cleans up the thread data structures.\\

    This path is executed within a new kernel thread. The major problem with this is
    synchronization. Occasionally a data structure is deleted and then accessed. This is
    a rare occurrence and we have been unable to track it down.

  \section{Userland Programs}
    \subsection{cat}
      The \code{cat} program functions like the UNIX equivalent. A filename is given as
      an argument and the contents of that file are sent to the console.
    
    \subsection{echo}
      The \code{echo} program is used to test argument passing. It's arguments are
      printed to the console.
    
    \subsection{fork}
      This program tests the \code{Fork()} system call. It prints a string, forks a new
      thread, and prints a second string. The new thread prints two strings. The
      \code{for()} loop in this program prevents a known problem from occurring, see
      section \ref{xref:knownproblems} for information on the \code{Fork/Exit} problem.
    
    \subsection{halt}
      This program halts the system.
    
    \subsection{procCount \& ps}
      These programs test the \code{GetProcCount()} and \code{GetProcInfo()} system calls.
    
    \subsection{touch}
      This program \code{Create}s a new file. This file will be empty.
    
    \subsection{shell}
      Our shell is a simple \code{Read}, \code{Parse}, \code{Exec} loop. A prompt is
      printed and the system waits for console input. Parsing splits up the resulting
      command line using \emph{space} as a delimiter. If the last ``argument'' is an
      ampersand (\&), then the resulting process is placed in the background. If a
      process is to be run in the foreground, after \code{Exec()} is called, \code{Join}
      is called on the returned PID. The shell can not deal with unnecessary white space.

    \subsection{others}
      There are more programs we used for testing various aspects of the system and for
      reliably recreating bugs during development.
